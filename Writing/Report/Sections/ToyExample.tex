Consider the incomplete preferences found in Table \ref{toy_incomplete}. Since most of our calculations are done with the complement of the ordinal preferences we also provide these in Table \ref{toy_complement}.

\begin{table}[h!]
\begin{center}
\caption{Incomplete preferences. Blank means unexpressed.}
\label{toy_incomplete}
\begin{tabular}{l|l|l|l|l|l}
\hline 
\textbf{Job Option} & \textbf{Seeker 1} &  \textbf{Seeker 2} & \textbf{Seeker 3} &\textbf{Seeker 4} &\textbf{Seeker 5} \\
\hline
\textbf{Job A} & 1 & 3 &  & 1 & 1\\
\textbf{Job B} & 2 & 2 & 1 & &\\
\textbf{Job C} & 3 & 1 &  & &\\
\hline 
\end{tabular}
\end{center}
\end{table}

\begin{table}[h!]
\begin{center}
\caption{Complement of incomplete preferences. Unexpressed preferences are replaced with 0.}
\label{toy_complement}
\begin{tabular}{l|l|l|l|l|l}
\hline 
\textbf{Job Option} & \textbf{Seeker 1} &  \textbf{Seeker 2} & \textbf{Seeker 3} &\textbf{Seeker 4} &\textbf{Seeker 5} \\
\hline
\textbf{Job A} & 2 & 0 & 0 & 2 & 1\\
\textbf{Job B} & 1 & 1 & 2 & 0 & 0\\
\textbf{Job C} & 0 & 2 & 0 & 0 & 0\\
\hline 
\end{tabular}
\end{center}
\end{table}

Now from here we can calculate the cosine similarity of each seeker to each other, this is in Table \ref{toy_similarity}.

\begin{table}[h!]
\begin{center}
\caption{Cosine Similarity between Job Seekers.}
\label{toy_similarity}
\begin{tabular}{l|l|l|l|l|l}
\hline 
\textbf{Job Seeker} & \textbf{Seeker 1} &  \textbf{Seeker 2} & \textbf{Seeker 3} &\textbf{Seeker 4} &\textbf{Seeker 5} \\
\hline
\textbf{Seeker 1} & 1 & $\frac{1}{5}$ & $\frac{1}{\sqrt{5}}$ & $\frac{2}{\sqrt{5}}$ & $\frac{2}{\sqrt{5}}$\\
\textbf{Seeker 2} & $\frac{1}{5}$ & 1 & $\frac{1}{\sqrt{5}}$ & 0 & 0 \\
\textbf{Seeker 3} &  $\frac{1}{\sqrt{5}}$ & $\frac{1}{\sqrt{5}}$ & 1 & 0 & 0 \\
\textbf{Seeker 4} & $\frac{2}{\sqrt{5}}$ & 0  & 0 & 1 & 1 \\
\textbf{Seeker 5} & $\frac{2}{\sqrt{5}}$ & 0  & 0 & 1 & 1\\
\hline 
\end{tabular}
\end{center}
\end{table}

We do a matrix multiplication of the complement of the expressed preferences with the similarity scores to get the interim preference score matrix $R$, found in Table \ref{toy_scores}.

\begin{table}[h!]
\begin{center}
\caption{Toy Matrix $R$. Known preferences have blanks for scores.}
\label{toy_scores}
\begin{tabular}{l|l|l|l|l|l}
\hline 
\textbf{Job Option} & \textbf{Seeker 1} &  \textbf{Seeker 2} & \textbf{Seeker 3} &\textbf{Seeker 4} &\textbf{Seeker 5} \\
\hline
\textbf{Job A} &  &  & $\frac{2}{\sqrt{5}}$ &  & \\
\textbf{Job B} &  &  &  &$\frac{2}{\sqrt{5}}$ &$\frac{2}{\sqrt{5}}$\\
\textbf{Job C} &  &  & $\frac{2}{\sqrt{5}}$ &0 & 0\\
\hline 
\end{tabular}
\end{center}
\end{table}

Finally, we use these scores to complete the preferences in Table \ref{toy_complete}. As you can see, we have a tie to break for Seeker 3. In this case, we simply choose with random uniform probability.

\begin{table}[h!]
\begin{center}
\caption{Completed Preferences.}
\label{toy_complete}
\begin{tabular}{l|l|l|l|l|l}
\hline 
\textbf{Job Option} & \textbf{Seeker 1} &  \textbf{Seeker 2} & \textbf{Seeker 3} &\textbf{Seeker 4} &\textbf{Seeker 5} \\
\hline
\textbf{Job A} & 1 & 3 & 3 & 1 & 1\\
\textbf{Job B} & 2 & 2 & 1 & 2 & 2\\
\textbf{Job C} & 3 & 1 & 2 & 3 & 3\\
\hline 
\end{tabular}
\end{center}
\end{table}
